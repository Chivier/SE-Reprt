\chapter{引言}
\section{编写目的}
在本项目的前一阶段,也就是需求分析阶段,已经将系统用户对本系统的需求做了详细的阐述,这些用户需求已经在上一阶段中对不同用户所提出的不同功能,实现的各种效果做了调研工作,并在需求规格说明书中得到详尽得叙述及阐明。

本阶段已在系统的需求分析的基础上,对即时聊天工具做概要设计。主要解决了实现该系统需求的程序模块设计问题。包括如何把该系统划分成若干个模块、决定各个模块之间的接口、模块之间传递的信息,以及数据结构、模块结构的设计等。在以下的概要设计报告中将对在本阶段中对系统所做的所有概要设计进行详细的说明,在设计过程中起到了提纲挈领的作用。

在下一阶段的详细设计中,程序设计员可参考此概要设计报告,在概要设计即时聊天工具所做的模块结构设计的基础上,对系统进行详细设计。在以后的软件测试以及软件维护阶段也可参考此说明书,以便于了解在概要设计过程中所完成的各模块设计结构,或在修改时找出在本阶段设计的不足或错误。


\section{项目背景}
随着互联网的快速发展和网速的不断提高,互联网用户的音乐需求已不再是单单的购买专辑并用专门的播放设备进行播放.他们更需要随时随地都能够收听到自己喜爱的歌曲,利用网络,而不是本地实体进行音乐的欣赏.同时,用户也迫切的希望能够定义一个属于自己的歌曲列表,并能够随时访问并更新.智能手机的产生为用户这一需求的满足提供了条件,本项目依托于云端存储和手机智能系统,为用户提供音乐点播功能,并允许用户自定义歌曲列表,以满足不同人群的风格需求.

\section{术语}
[列出本文档中所用到的专门术语的定义和外文缩写的原词组]
\begin{table}[htbp]
\centering
\caption{术语表} \label{tab:terminology}
\begin{tabular}{|c|c|}
    \hline
    缩写、术语 & 解释 \\
    \hline
    SQL & SQL 指结构化查询语言,使我们有能力访问数据库 \\
    \hline
\end{tabular}
% \note{这里是表的注释}
\end{table}