\chapter{出错处理设计}
\section{数据库出错处理}
当保存用户信息的数据库出错时,暂停整个用户系统的服务,涉及到用户信息的所有操作向客户端返回报错响应.

当歌曲数据库发生错误时,暂停整个系统的服务,对用户发送的所有搜索,播放,下载请求返回报错响应.

当专辑,歌手数据库发生错误时,对于一切进入专辑/歌手界面的操作返回报错相应.

当歌单数据库发生错误时,对于一切更新云端歌单数据库的请求返回报错相应,此时客户端应使用本地维护的歌单数据库.

在数据库发生错误时,首先尝试使用日志恢复到正常状态;如果不行,尝试恢复本地保存的备份数据;如果本地备份数据也已损坏,尝试恢复容灾节点的备份数据.如果以上操作均失败,则云端自行恢复数据库已变的不可能,向客户发出服务已失效的公告.对于歌单数据库发生的错误,考虑以下解决方案:
在已经告知用户云端数据库已毁坏的情况下,由客户端向云端发送保存在本地的歌单数据,以最大程度的恢复数据.

\section{某模块失效处理}

是否整个系统暂停服务,还是维持最小服务状态、如何尽快恢复服务还是删库跑路等。
\subsection{本地模块失效处理}

\subsubsection{下载模块}

当下载模块失效时,暂停下载模块的工作,请求模块不再发送下载请求,当确定模块失效是广泛存在时,由开发者对bug进行修复,并尽快推送更新给所有用户.

\subsubsection{音频播放模块}

当音频播放模块失效时,暂停所有播放功能,其他模块正常工作,当确定模块失效是广泛存在时,由开发者对bug进行修复,并尽快推送更新给所有用户.

\subsubsection{请求发送/接收模块}

当请求发送/接收模块失效时,暂停下载模块的工作,其他模块正常工作,首先由本地确定失效原因,如果确定失效原因由代码内部的错误引起,则不断尝试向云端发送错误信息.当确定模块失效是广泛存在时,由开发者对bug进行修复,并尽快推送更新给所有用户.

\subsubsection{本地数据库维护模块}

当本地数据库维护模块失效时,暂停所有模块的工作,尝试清空本地数据库.如果清空数据库后本地数据库维护仍处于失效状态,则认定该应用已不可用,向用户发出错误信息,向用户建议重新安装应用程序.当确定模块失效是广泛存在时,由开发者对bug进行修复,并尽快推送更新给所有用户.

\subsubsection{界面显示模块}

当界面显示模块失效时,暂停所有模块的工作,认定该应用已不可用,向用户建议重新安装应用程序.当确定模块失效是广泛存在时,由开发者对bug进行修复,并尽快推送更新给所有用户.

\subsection{云端模块失效处理}

\subsubsection{后台数据库维护模块}

当后台数据库维护模块失效时,实行上一节介绍的数据库出错处理操作,恢复数据库数据.如果后台数据库维护模块的失效不是由数据错误引起的,则尽快对代码bug进行修复.

后台数据库维护模块失效时,一切发往云端的请求都将收到一个保存响应,同时暂停云端上其他模块的工作.

\subsubsection{云端请求/回应处理模块}

当云端请求/回应处理模块失效时,暂停相似推荐模块的工作,尽快对代码bug进行修复.

\subsubsection{相似推荐模块}

当相似推荐模块失效时,在云端请求/回应处理模块对所有相似推荐的请求返回报错相应,尽快对代码bug进行修复.
