\chapter{数据结构设计}
\section{逻辑结构设计}
\subsection{用户管理系统数据结构设计}
讲述本系统内需要什么数据结构。这指的是程序运行过程中维护的数据结构。只是举个例子,此处应和3.3一致。
\subsection{客户端数据结构}

1.下载模块:

维护数据结构:下载源字符串类型,缓冲区,通信计时

客户端的下载模块需要维护一个字符串类型来存储服务端返回的下载源,下载模块调用本地接口请求目标源的数据包,下载模块并需要动态缓冲区类型来从下载源存储正在下载的数据,数据结构通信计时器查看是否发生请求超时或者丢包等情况影响下载。



2.请求发送/接收模块:

请求接收模块是一个基本的重要模块,它需要维护数据结构有: 请求/接收包数据规模,数据缓冲区,通信计时,校验串

请求、接收模块是对数据请求和响应的发送与处理模块,数据缓冲区用来存储数据包的内容,通信计时器和校验串为了保证通信的正常进行和数据包的完整性,动态变化的请求/接收数据规模数据结构设定为为数据缓冲区分配存储空间的大小。

3.音频播放模块(包括流媒体):

维护的数据结构: 音频数据缓冲区,进度参数,计时器

根据云端发回的音频数据,客户端需要将音频数据载入音频数据缓冲区,进度参数为播放该歌曲的进度,可由前端用户更改跳转至对应的音频数据缓冲区位置,计时器为该歌曲所播放的时间随播放进行动态变化。



4.本地数据库维护模块:

维护数据结构:用户ID和权限,与云端同步的标志位信息,数据库操作类型

根据当前登录的用户ID和其拥有的权限可拥有对本地数据的一定访问权限,根据与云端同步的标志位信息判断该维护动作是否合法,并利用数据库操作类型选定需要进行的数据库动作。

5.界面显示模块:
维护数据结构:界面结构信息、各结构块规模和结构体内容

根据设定的各界面结构信息,各结构体块规模,配合结构体内容向用户展示对应的界面窗口。

\subsection{服务端数据结构}

后端数据库维护模块:

数据结构:

数据库操作:
\begin{itemize}
	\item 操作表对象;
	\item 操作类型
	\item 操作值
	\item 操作语句
	\item 发生时间
	\item 是否成功
\end{itemize}




表对象信息:
\begin{itemize}
	\item 表名
	\item 表记录数
	\item 表最后一次操作
	\item 表可用性
\end{itemize}


云端请求/回应:
\begin{itemize}
	\item 请求来源
	\item 目标URL
	\item 请求类型
	\item 请求资源
	\item 请求源代码
	\item 发生时间
	\item 是否已回应
\end{itemize}



回应:{
\begin{itemize}
	\item 回应目标URL
	\item 回应来源
	\item 回应信息体
	\item 回应对应请求
	\item 回应时间
	\item 是否已送达
\end{itemize}


相似推荐模块:
\begin{itemize}
	\item 歌曲id
	\item 歌曲名称
	\item 歌曲标签
\end{itemize}



\section{物理结构设计}
无
