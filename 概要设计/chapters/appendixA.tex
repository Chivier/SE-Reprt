\chapter{可行性分析结果}
\section{进行可行性分析的方法}
本次可行性分析复查项目目标和规模,研究目前正使用的系统,导出新系统的高层逻辑模型,重新定义问题。

\section{评价尺度}
本系统进行评价时的主要尺度有:费用的多少,开发时间的长短,以及使用的难易程度等。
\section{可行性分析过程}
\subsection{技术条件可行性分析}
本系统是一个采用云存储的音乐播放与推荐系统,采用面向对象技术、数据库技术、分布式技术等先进技术开发的应用程序,现有的开发技术已非常成熟,且被广泛应用于各行各业,利用现有技术完全可以达到功能目标。考虑开发期限较为充裕,预计可以在规定的时间内完成开发。  
\subsection{经济可行性分析}
\subsubsection{支出}
\paragraph{基本建设投}
\begin{enumerate} 
	\item 硬件设备:服务器。  
	\item 软件:Windows 2000 Server 或 Linux、数据库管理系统:SQL Server。
	\item 开发工具:Eclipse。 
	\item 软件平台:Tomcat。
\end{enumerate} 
\paragraph{其他一次性支出}
系统设计和开发费用。 
\paragraph{非一次性支出}
系统维护费用。 
\subsubsection{收益}
系统部署后自动化的操作,减少了人力、物力费用,极大地提高了工作效率和系统性能。 
\subsubsection{投资回报周期}
根据投资回收期计算方法,收益的累计数开始超过支出的累计数的时间为1年
\subsubsection{社会因素方面的可行性}
\paragraph{法律方面的可行性}

所建议系统的研制和开发都选用正版软件,云音乐采用不在版权周期的音乐,将不会侵犯他人、集体和国家的利益,不会违反相关的国家政策和法律。 

\paragraph{操作方面的可行性}
本系统的研制和开发充分考虑用户工作流程、计算机操作水平等,尽可能提供更人性化、直观的界面,满足用户要求。系统的操作方式在用户组织内可行。 
\section{可行性的结论}
经上述可行性分析,系统的研制和开发可以立即开始进行