\chapter{具体需求}
\section{功能需求}

本子章节描述软件产品的输入怎样被转换成输出。它描述了软件必须执行的基本动作。 



\subsection{R.INTF.CLD.001 搜索}
\paragraph{介绍}
用户可以通过搜索框向云端发出搜索请求,云端从曲库中搜索相关歌曲,专辑,歌单,歌手信息,分别整理为列表并返回给本地客户端.

\paragraph{输入}
输入来源:本地客户端

输入格式:Unicode字符串

有效输入范围:1-99个字符

\paragraph{处理}

输入的有效性检测:搜索字符串应为长度为1-99的Unicode字符串


操作时序:
\begin{enumerate}
	\item 用户在搜索框内键入搜索字符串
	\item 用户点击搜索按钮向云端提交搜索请求
	\item 云端对搜索字符串进行验证,确定其是合法字符串
	\item 云端在歌曲,专辑,歌手,歌单 数据库中分别应用字符串进行搜索并整理成列表
	\item 云端将结果发回至客户端
	\item 客户端收到结果后,将其以合适的形式显示到屏幕上
\end{enumerate}

异常处理:
\begin{enumerate}
	\item 客户端输入框内字符串不合法:点击搜索按钮时客户端报告不合法输入,不发出搜索请求
	\item 云端接收到的字符串不合法:使搜索结果为空并返回到客户端
	\item 云端和本地通信失败:客户端提示失败原因,并进行收到空的结果的行为
\end{enumerate}


使用方法:

对数据库内执行搜索的算法:模糊搜索(待定)




\paragraph{输出}

从云端到本地:

输出位置:客户端请求接收器
输出数量:4
输出单位:信息列表
具体描述:4个信息列表分别信息单位为歌曲,专辑,歌手,歌单

从结果到屏幕:
输出位置:客户端屏幕
输出数量:4
具体描述:客户端通过选择结果标签页确定要显示歌曲,专辑,歌手,歌单的其中一个列表

\subsection{R.INTF.CLD.002 播放歌曲}

\paragraph{介绍}

播放是云音乐系统的最基本功能,用户通过选中界面上的一个歌曲项,即可播放相应歌曲.音乐文件的存储位置有本地和云端两个选项,用户可以通过下载功能来将云端的音乐文件下载到本地,在播放时将直接播放本地的歌曲.如果音乐文件不在本地,则云端通过流媒体的形式返回给客户端一个流,客户端再播放这个流.

\paragraph{输入}

输入来源:客户端
输入内容:一个歌曲项

用户选中界面上的一个歌曲项,歌曲项是客户端歌曲组织的基本形式,用户可以直接点击歌曲项来执行播放功能.用户可以从歌曲项获得歌曲的名称,歌手,所属专辑等信息.歌曲项还包含了一个独一无二的编码用来区分并永久性保存.事实上,发送到云端的一个播放请求可以仅是这个编码

\paragraph{处理}

有效性检测:

有效性的检测包含两个方面:本地和云端,本地的有效性检测优先于云端并是一个短路操作.如果在本地,歌曲被验证是有效的,即歌曲已在本地,则直接播放,不与云端交互.当本地无法验证有效性时,云端将检测请求播放的曲目是否有效,在本系统中等价于歌曲是否在云端曲库中,如果不在,播放请求无效.否则,认定播放请求有效.

操作时序:
\begin{enumerate}
	\item 用户点击客户端的歌曲项进行播放操作
	\item 系统生成播放请求,并在本地检测请求对应的歌曲是否在本地,如果在,直接播放该歌曲并进行相应状态的更新.否则,进入下一步.
	\item 客户端向云端发出请求,云端接收请求后检查该请求是否合法,如果不合法,向本地发回不合法原因.否则,云端创建一个流返回给客户端.
	\item 客户端接收到云端发来的流,进行播放,并进行相应状态的更新.
\end{enumerate}


异常处理:
\begin{enumerate}
	\item 云端不存在请求歌曲(请求不合法):向客户端发回不合法回应,指出该歌曲不存在
	\item 播放流时网络中断:播放进行到已缓存的最后位置停止,显示"网络已中断"信息
\end{enumerate}


\paragraph{输出}

输出目标:客户端
输出内容:流/媒体文件

只要请求合法,客户端总能获得一个媒体文件/流,客户端使用相应的api播放这个媒体文件/流


\subsection{R.INTF.CLD.003 收藏}

\paragraph{介绍}

收藏功能是云音乐系统的基本功能之一.用户通过收藏功能,可以对歌曲/专辑/歌单/歌手进行标记,可以在以后快速的找到相关信息.用户的收藏信息是用户个性化的一部分,其数据将用于推荐功能中.同时,用户的收藏信息也将保存在云端,和客户端保持同步.对于收藏功能,在这里以歌曲为进行介绍,对于专辑/歌单/歌手的收藏,其行为表现与歌曲是一致的,对于歌曲独有的特性,会做出重点说明.

收藏操作是一个同步操作,要求用户在联网的状态下完成,否则视为一个异常进行处理.

\paragraph{输入}

输入来源:客户端
输入内容:歌曲+目标歌单

在客户端界面,用户可以点击歌曲项的菜单按钮进行收藏操作,用户需要手动选择希望保存到的歌单.对于"我最喜爱的歌曲"可以直接点击歌曲项上的快捷按钮,这时默认目标歌单为:我最喜爱的歌曲

\paragraph{处理}

有效性检测:

有效性检测包含歌曲和歌单两个有效性的检测:
歌曲有效性:目标歌曲必须在云端曲库中
歌单有效性:目标歌单必须在云端中并且所有者为发出请求的用户
以上两个任何一个无效都认为该次请求无效,返回无效的响应

操作时序:
\begin{enumerate}
	\item 用户在客户端进行收藏操作,客户端封装呈请求发送到云端
	\item 云端进行有效性检测,如果请求被检测为无效的,则返回给客户端无效的响应;否则,进入下一步.
	\item 系统将目标歌曲加入到目标歌单列表中,并返回给客户端 操作成功 的相应.
	\item 客户端收到相应,并在本地对数据库进行更新,更新缓存在本地的歌单列表
\end{enumerate}


异常处理:
\begin{enumerate}
	\item 在发送请求时网络中断:本次收藏操作无效,并显示错误信息.
	\item  无法收到云端发回的响应:令客户端不对歌单进行更新,直到下一次与云端同步.
\end{enumerate}


\paragraph{输出}

输出目标:客户端
输出内容:来自云端的请求响应

客户端收到云端的请求响应后,进行相应的本地歌单更新操作

\subsection{R.INTF.CLD.004 客户端}

客户端是用户与本系统交互的重要途径。本客户端具体实现是一个安卓系统上的app应用程序。客户端提供了用户完成其他需求的接口。用户可以简单的使用这个app程序而完成搜索、播放音乐、收藏音乐、使用推荐系统、完善个人信息等其他操作。

\subsubsection{需求1:前端设计}

\paragraph{介绍}

前端ui是客户端提供给用户的接口,因此前端的设计要有足够的逻辑性和方便性,这样可以方便用户进行各种操作,诸如搜索、播放音乐、收藏音乐、使用推荐系统、完善个人信息等。这些操作在前端中对应的操作应当是简洁明了的,这样才能给用户良好的体验。同时,前端还为后端提供了标准化的接口,以告知后端用户进行了哪些操作;这个接口也可以让后端将计算结果和服务器中下载的内容有一个接口用以反馈给用户。这些接口的定义应当具有足够的向上兼容性,以满足之后几代的客户端都可以使用类似的接口进行代码操作。

\paragraph{输入}

前端输入数据包括用户的操作和后端发送来的处理后的数据

\paragraph{处理}

对于这些数据,前端起到转发的作用。将用户操作数据直接发送给后端,将后端数据直接发送到屏幕显示

\paragraph{输出}

屏幕显示和后端接口


\subsubsection{需求2:后端设计}

\paragraph{介绍}

后端实现了各种后台运算和对服务器远程连接的支持。后台运算的需求包括可以将音乐文件高质量的解析出来并且调用系统的音效设备将音乐播放出来;用户使用此系统听音乐时后台不断进行这样的操作,这要求后端的音乐解析算法在保证解析质量的前提下足够高效。后端同时也要实现与服务器的通讯。后端应当实时感知系统的网络状态,在网络状态符合要求的情况下,后端将客户搜索歌曲、下载歌曲、填写个人数据等请求发送给服务器,并且接受服务器的回应。当接受到服务器压缩发送的信息后,后端对这些信息适当操作使其成为能够被前端解析的数据,并发送给前端。

\paragraph{输入}

前端发来的用户操作数据、
服务器发来的数据

\paragraph{处理}

用户操作数据作为激励信号,使后端运行对应的函数或者过程。
服务器发来的数据将在后端被解析,使之成为能够被前端识别的数据格式

\paragraph{输出}

传送到服务器的数据包。传送到前端的可以直接使用的数据

\subsection{R.INTF.CLD.005 个人信息}
个人信息由用户填写或者生成并上传服务器。采集这些信息有助于我们更好的服务用户,并且对用户提供有针对性的高质量服务。
\subsubsection{需求1:用户授权的需求}
\paragraph{介绍}
此系统采集用户信息以完善服务的过程将采集用户的一部分隐私数据,因此首先的需求是要有用户的授权。这里采用让用户使用此系统前首先确认同意此系统的服务协议的方式获得用户的授权。
\paragraph{输入}
输入来源:
用户同意用户协议的操作

数量和度量单位:
一次用户操作

时间要求:
无

包含精度和容忍度的有效输入范围:
要求精度高,即没有确认的用户不会因为误报而造成后台以为其同意的错误
\paragraph{处理}
服务器得到用户授权同意的信息后将在服务器中创建对应的账户,记录用户的信息,提供个性化服务。
\paragraph{输出}
开始提供个性化服务

\subsubsection{需求2:用户登陆信息}
\paragraph{介绍}
用户登陆信息是用于识别用户身份认证的信息。包括用户名和密码。

\paragraph{输入}
输入来源:
由于用户输入

数量:
仅几个字符

时间要求:
在用户可以忍受的时间内(1min)识别并确认该用户

包含精度和容忍度的有效输入范围:
精度要求极高,当用户输入有错时直接反馈错误情况。

\paragraph{处理}
用户名的需求是可替代性和冗余性。比如用户邮箱、音乐账号、手机号、QQ号这些信息都可以作为同一用户实体的不同用户名。我们要妥善管理这些信息,使同一用户的不同登陆信息能够识别出同一用户。而密码则是确保用户为本人操作。此云音乐系统中只需要第一次登陆的时候输入密码,此后每次登陆的时候就有记住密码的功能。这一需求是将密码以一个加密的存储在本地。密码同时也有保密性的需求,我们的数据中心服务器、用户本地都不会存储明文密码,只会存储加密后的密文。同时密码的传输会使用SSH加密传输,来保证密码不被窃取。

用户兴趣信息由用户填写或者生成。由用户填写的部分包括用户的年龄、性别、学历、喜爱的歌手等。而由用户生成的信息包括用户听歌记录、收藏的歌单、关注的用户等。这些用户的兴趣信息主要是用来为本系统的个性化推荐系统服务的。考虑到这些信息也涉及用户的部分隐私,此类信息传输时依然使用加密方式传输。而这些信息在中心服务器使用推荐算法分析时,将首先被去标签化处理。
这些信息包括和用户兴趣信息这两个部分。

\subsection{R.INTF.CLD.006 动态主页}
动态主页实时反映用户的个人信息,包括听歌记录、兴趣记录,也包括用户自定义的信息,如用户名、个性签名、头像等。
\subsubsection{需求1:实时性}
\paragraph{介绍}
实时性的需求要求动态主页能够实时更新,反应出用户的最新兴趣情况和使用记录。
\paragraph{输入}
输入来源:
实时性的更新数据输入来自软件的实时追踪用户使用情况和实时向服务器发送用户信息。这样才服务器得到获取动态主页去请求时,服务器始终能提供最新版本的动态主页数据。

数量:
若实时采集这些数据,其数量将十分庞大。而本系统将在客户端首先进行特征提取和压缩的操作,因此反馈到服务器端的数据量将会是很小的,一般只有几kB的数量。

度量单位:
实时性的度量单位是更改发生后要延迟才能体现在动态主页上。

时间要求:
实时性本身就是对时间的要求。

包含精度和容忍度的有效输入范围:
对于用户填写的内容。实时性要求精度高。而对于用户生成的内容,这些内容反应是是用户的兴趣变化实时性精度要求不高。

\paragraph{处理}
服务器端会根据这些信息更新该用户在服务器端的信息。并且在有新的访问信息的请求时提供新版本的信息

\paragraph{输出}
用户信息的实时变化。







\section{性能需求}


这里介绍了性能方面的需求和他们的原理。以帮助开发者理解意图以做出正确的设计选择。在实时系统中的时序关系也列在其中.
\subsection{R.PF.CLD.001 网络并发性}

最大可同时支持流媒体连接:1000个320Kbps码率的流媒体

最大可同时支持用户在线数:5000个



\subsection{R.PF.CLD.002 延迟}

1.所有用户发出的任何网络请求应在5秒内得到云端的响应,这里以用户从客户端实际发出请求的时间为准.

2.用户登录的操作优先级应高于其他所有操作.


\subsection{R.PF.CLD.003 连接速度}

1. 在用户网络条件允许的情况下,应保证95\%上的流媒体连接保持100KBps以上的速度.

2. 在用户网络条件允许的情况下,应保证95\%上的下载连接保持50KBps以上的速度.

\subsection{R.PF.CLD.004 数据库容量}

在任何情况下,曲库都可以容纳不超过以下大小的数据:

1.10000首码率为320Kbps,时长为6min的MP3格式的歌曲

2.1000名歌手的数据

3.1000枚专辑的数据

4.10000名用户的数据


\subsection{R.PF.CLD.005 时间精确度}

用户对流媒体播放的时间定位应精确到秒,实际返回的流起始位置应在用户指定位置为中心的1秒区间内. 
 
\section{外部接口需求}

\subsection{Android系统应用}

最低Android系统版本:4.4

支持屏幕清晰度:xhpi以上

支持功能:上述功能需求中的所有功能

在应用主界面,用户应当可以看到一个明显的导航栏,分别指向我的音乐和动态音乐两个页面,同时,用户也可以通过拖拽侧边栏的方式进入个人信息的管理导航栏中.
应用的进入界面是动态主页页面,本页面包含推荐歌单项和新音乐,新专辑项目,用户通过点击相关项目可以进入相应页面.
在我的音乐界面中,包含我的收藏,我的歌单两个分类.用户通过点击我的收藏项目,可以进入列表视图,包含专辑,歌手项目,用户可以在此找到自己收藏的项目.在我的歌单中,列出了用户创建或收藏的歌单,其中"我最喜爱的歌单"总在列表的第一项.

每个Android终端都包含back,home,menu三个按键,功能分别如下:
honme:返回Android系统界面
back:返回上一个活动界面,如果该界面已不可用,则返回主界面
menu:点击:以Android系统设置为准;长按:打开个人设置侧边栏.

用户在使用三次以上app时,应当能够找到我最喜爱的音乐,推荐歌单的位置,并可以完成播放,搜索,下载等工作.

用户不应该在客户端内完成用户邮箱,密码等安全信息的更改,而应该在系统提供的web页面处进行.




\subsection{软件接口}

\subsubsection{R.SI.CLD.001 歌曲项}

歌曲项是在系统内进行歌曲信息交换的核心结构,其应用于多个接口,该结构包含以下内容:

1.歌曲名:1-99字符的Unicode字符串

2.歌曲编码:1-99字符的Unicode字符串,唯一

3.所属专辑:1-99字符的Unicode字符串

4.歌手/团体:1-99字符的Unicode字符串

5.发布时间:由数据库系统定义的日期变量

在请求中,可以将歌曲项简化为歌曲编码进行传输,云端通过编码确定其对应的歌曲项,并将其拓展为完整的歌曲项返回给客户端.

\subsubsection{R.SI.CLD.002 专辑项}

专辑项是在系统内进行专辑信息交换的核心结构,其应用于多个接口,该结构包含以下内容:

1.专辑名:1-99字符的Unicode字符串

2.专辑编码:1-99字符的Unicode字符串,唯一

3.所属歌手/团体:1-99字符的Unicode字符串

4.发布时间:由数据库系统定义的日期变量

在请求中,可以将专辑项简化为专辑编码进行传输,云端通过编码确定其对应的歌曲项,并将其拓展为完整的专辑项返回给客户端.

\subsubsection{R.SI.CLD.003 歌手/团体项}

歌手/团体项是在系统内进行歌手交换的核心结构,一首歌曲,一枚专辑可以属于一名歌手,也可以属于一个团体,但考虑到歌手可以视作为一个成员只有其自身的团体,故在系统内对歌手/团体不作区分

1.歌手/团体名:1-99字符的Unicode字符串,唯一

\subsubsection{R.SI.CLD.004 搜索系统与数据库系统}

搜索系统需要向数据库系统发出搜索请求,数据库系统在其所有的数据库中进行查询操作后,返回若干个结果列表.

参数:合法的搜索字符串
返回值:若干个结果列表,列表的数目由搜索类型的数目确定.

\subsubsection{R.SI.CLD.005 查找本地歌曲文件}

客户端在收到播放一个指定歌曲的曲目时,获取该歌曲编码并拓展成一个全局唯一的歌曲文件名,在操作系统目录下搜索该媒体文件并播放.

参数:歌曲文件名
返回值:一个歌曲文件;或为空,如果歌曲文件不存在.

\subsubsection{R.SI.CLD.006 登录}

客户端向云端发送登陆请求,请求中包含用户名及密码,云端对其进行验证后向客户端发送登录结果.

参数:用户名,密码
返回值:登陆结果(成功/失败)




\subsection{硬件接口} 

本系统的客户端支持一切运行在Android4.4以上系统的智能手机上,其客户端的函数接口定义在Andoird SDK中,其接口已在其文档中详细描述,在此不再介绍.

\subsection{通讯接口}

\subsubsection{R.CI.CLD.001 TCP/IP}

本地客户端和云端之间所有通信都通过TCP/IP协议完成,该协议的具体说明在其定义文档中已清楚描述,在此不作说明.

\subsubsection{R.CI.CLD.002 HTTP}

本地客户端和云端之间的所有通信都通过HTTP协议完成,该协议的具体说明在其定义文档中已清楚描述,在此不作说明.

