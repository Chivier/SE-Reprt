\chapter{任务概述}
本系统的目标是实现一个云音乐播放系统,包括客户端、服务器端两个部分。

客户端面向个人用户,为用户提供音乐播放、音乐管理和音乐推荐服务。

\section{目标}
实现云音乐播放系统系统,实现需求规格说明书中所描述的
歌曲播放 (本地/在线)功能、
歌曲搜索功能、
歌单创建与收藏功能、
关联推荐、
每日推荐歌单、
用户个人信息管理 (登录, 注册, 信息更改)功能
并且保证系统的健壮性和数据安全。

\section{开发与运行环境}

\subsection{开发环境的配置}
\begin{table}[htbp]
	\centering
	\caption{开发环境的配置} \label{tab:development-environment}
	\begin{tabular}{|c|c|c|}
		\hline
		类别 & 标准配置 & 最低配置 \\
		\hline
		计算机硬件 & \tabincell{c}{基于x86结构的CPU\\ 主频>=2.4GHz\\ 内存>=8G\\ 硬盘>=200G } & \tabincell{c}{基于x86结构的CPU\\ 主频>=1.6GHz\\ 内存>=512M\\ 硬盘>=2G} \\
		\hline
		计算机软件 & \tabincell{c}{Linux (kernel version>=4.10)\\ GNU gcc (version>=6.3.1)\\Android Studio 2.3.1 } & \tabincell{c}{Linux (kernel version>=3.10)\\ GNU gcc (version>=5.4\\Android Studio 2.3.1} \\
		\hline
		网络通信 & \tabincell{c}{至少要有一块可用网卡\\ 能运行IP协议栈即可} & \tabincell{c}{至少要有一块可用网卡\\ 能运行IP协议栈即可} \\
		\hline
		其他 & \tabincell{c}{采用 MySQL 数据库\\ 使用git进行版本控制} & \tabincell{c}{采用 MySQL 数据库\\ 使用git进行版本控制} \\
		\hline
	\end{tabular}
	% \note{这里是表的注释}
\end{table}

\subsection{测试环境的配置}
\begin{table}[htbp]
	\centering
	\caption{服务器端测试环境的配置} \label{tab:test-environment}
	\begin{tabular}{|c|c|c|}
		\hline
		类别 & 标准配置 & 最低配置 \\
		\hline
		计算机硬件 & \tabincell{c}{基于x86结构的CPU\\ 主频>=2.4GHz\\ 内存>=8G\\ 硬盘>=200G } & \tabincell{c}{基于x86结构的CPU\\ 主频>=1.6GHz\\ 内存>=512M\\ 硬盘>=2G} \\
		\hline
		计算机软件 & \tabincell{c}{Linux (kernel version>=4.10)\\ GNU gcc (version>=6.3.1)\\Android Studio 2.3.1 } & \tabincell{c}{Linux (kernel version>=3.10)\\ GNU gcc (version>=5.4)\\Android Studio 2.3.1} \\
		\hline
		网络通信 & \tabincell{c}{至少要有一块可用网卡\\ 能运行IP协议栈即可} & \tabincell{c}{至少要有一块可用网卡\\ 能运行IP协议栈即可} \\
		\hline
		其他 & \tabincell{c}{采用 MySQL 数据库\\ 使用git进行版本控制} & \tabincell{c}{采用 MySQL 数据库\\ 使用git进行版本控制} \\
		\hline
	\end{tabular}
	% \note{这里是表的注释}
\end{table}

\begin{table}[htbp]
	\centering
	\caption{客户端测试环境的配置} \label{tab:test-environment}
	\begin{tabular}{|c|c|c|}
		\hline
		类别 & 标准配置 & 最低配置 \\
		\hline
		计算机硬件 & \tabincell{c}{基于ARM结构的CPU\\ 主频>=2.0GHz\\ 内存>=2G\\ 硬盘>=16G} & \tabincell{c}{基于ARM结构的CPU\\ 主频>=1.6GHz\\ 内存>=1M\\ 硬盘>=2G} \\
		\hline
		计算机软件 & \tabincell{c}{Android (version>=6.1)\\ Android Studio 2.3.1 } & \tabincell{c}{Android (version>=4.1)\\ Android Studio 2.3.1 } \\
		\hline
		网络通信 & \tabincell{c}{至少要有一块可用网卡\\ 能运行IP协议栈即可} & \tabincell{c}{至少要有一块可用网卡\\ 能运行IP协议栈即可} \\
		\hline
		其他 & \tabincell{c}{能进行音乐播放} & \tabincell{c}{能进行音乐播放} \\
		\hline
	\end{tabular}
	% \note{这里是表的注释}
\end{table}


\subsection{运行环境的配置}
\begin{table}[htbp]
	\centering
	\caption{服务器端测试环境的配置} \label{tab:test-environment}
	\begin{tabular}{|c|c|c|}
		\hline
		类别 & 标准配置 & 最低配置 \\
		\hline
		计算机硬件 & \tabincell{c}{基于x86结构的CPU\\ 主频>=2.4GHz\\ 内存>=8G\\ 硬盘>=200G } & \tabincell{c}{基于x86结构的CPU\\ 主频>=1.6GHz\\ 内存>=512M\\ 硬盘>=2G} \\
		\hline
		计算机软件 & \tabincell{c}{Linux (kernel version>=4.10)\\ GNU gcc (version>=6.3.1)\\Android Studio 2.3.1 } & \tabincell{c}{Linux (kernel version>=3.10)\\ GNU gcc (version>=5.4)\\Android Studio 2.3.1} \\
		\hline
		网络通信 & \tabincell{c}{至少要有一块可用网卡\\ 能运行IP协议栈即可} & \tabincell{c}{至少要有一块可用网卡\\ 能运行IP协议栈即可} \\
		\hline
		其他 & \tabincell{c}{采用 MySQL 数据库\\ 使用git进行版本控制} & \tabincell{c}{采用 MySQL 数据库\\ 使用git进行版本控制} \\
		\hline
	\end{tabular}
	% \note{这里是表的注释}
\end{table}

\begin{table}[htbp]
	\centering
	\caption{客户端测试环境的配置} \label{tab:test-environment}
	\begin{tabular}{|c|c|c|}
		\hline
		类别 & 标准配置 & 最低配置 \\
		\hline
		计算机硬件 & \tabincell{c}{基于ARM结构的CPU\\ 主频>=2.0GHz\\ 内存>=2G\\ 硬盘>=16G} & \tabincell{c}{基于ARM结构的CPU\\ 主频>=1.6GHz\\ 内存>=1M\\ 硬盘>=2G} \\
		\hline
		计算机软件 & \tabincell{c}{Android (version>=6.1) } & \tabincell{c}{Android (version>=4.1) } \\
		\hline
		网络通信 & \tabincell{c}{至少要有一块可用网卡\\ 能运行IP协议栈即可} & \tabincell{c}{至少要有一块可用网卡\\ 能运行IP协议栈即可} \\
		\hline
		其他 & \tabincell{c}{能进行音乐播放} & \tabincell{c}{能进行音乐播放} \\
		\hline
	\end{tabular}
	% \note{这里是表的注释}
\end{table}

\section{需求概述}
功能需求包括:

\subsection{R.INTF.CLD.001 搜索}
\paragraph{介绍}
用户可以通过搜索框向云端发出搜索请求,云端从曲库中搜索相关歌曲,专辑,歌单,歌手信息,分别整理为列表并返回给本地客户端.

\paragraph{输入}

输入来源:本地客户端

输入格式:Unicode字符串

有效输入范围:1-99个字符

\paragraph{处理}

输入的有效性检测:搜索字符串应为长度为1-99的Unicode字符串


操作时序:
\begin{enumerate}
	\item 用户在搜索框内键入搜索字符串
	\item 用户点击搜索按钮向云端提交搜索请求
	\item 云端对搜索字符串进行验证,确定其是合法字符串
	\item 云端在歌曲,专辑,歌手,歌单 数据库中分别应用字符串进行搜索并整理成列表
	\item 云端将结果发回至客户端
	\item 客户端收到结果后,将其以合适的形式显示到屏幕上
\end{enumerate}

异常处理:
\begin{enumerate}
	\item 客户端输入框内字符串不合法:点击搜索按钮时客户端报告不合法输入,不发出搜索请求
	\item 云端接收到的字符串不合法:使搜索结果为空并返回到客户端
	\item 云端和本地通信失败:客户端提示失败原因,并进行收到空的结果的行为
\end{enumerate}


使用方法:

对数据库内执行搜索的算法:模糊搜索(待定)




\paragraph{输出}

从云端到本地:

输出位置:客户端请求接收器
输出数量:4
输出单位:信息列表
具体描述:4个信息列表分别信息单位为歌曲,专辑,歌手,歌单

从结果到屏幕:
输出位置:客户端屏幕
输出数量:4
具体描述:客户端通过选择结果标签页确定要显示歌曲,专辑,歌手,歌单的其中一个列表

\subsection{R.INTF.CLD.002 播放歌曲}

\paragraph{介绍}

播放是云音乐系统的最基本功能,用户通过选中界面上的一个歌曲项,即可播放相应歌曲.音乐文件的存储位置有本地和云端两个选项,用户可以通过下载功能来将云端的音乐文件下载到本地,在播放时将直接播放本地的歌曲.如果音乐文件不在本地,则云端通过流媒体的形式返回给客户端一个流,客户端再播放这个流.

\paragraph{输入}

输入来源:客户端

输入内容:一个歌曲项

用户选中界面上的一个歌曲项,歌曲项是客户端歌曲组织的基本形式,用户可以直接点击歌曲项来执行播放功能.用户可以从歌曲项获得歌曲的名称,歌手,所属专辑等信息.歌曲项还包含了一个独一无二的编码用来区分并永久性保存.事实上,发送到云端的一个播放请求可以仅是这个编码

\paragraph{处理}

有效性检测:

有效性的检测包含两个方面:本地和云端,本地的有效性检测优先于云端并是一个短路操作.如果在本地,歌曲被验证是有效的,即歌曲已在本地,则直接播放,不与云端交互.当本地无法验证有效性时,云端将检测请求播放的曲目是否有效,在本系统中等价于歌曲是否在云端曲库中,如果不在,播放请求无效.否则,认定播放请求有效.

操作时序:
\begin{enumerate}
	\item 用户点击客户端的歌曲项进行播放操作
	\item 系统生成播放请求,并在本地检测请求对应的歌曲是否在本地,如果在,直接播放该歌曲并进行相应状态的更新.否则,进入下一步.
	\item 客户端向云端发出请求,云端接收请求后检查该请求是否合法,如果不合法,向本地发回不合法原因.否则,云端创建一个流返回给客户端.
	\item 客户端接收到云端发来的流,进行播放,并进行相应状态的更新.
\end{enumerate}


异常处理:
\begin{enumerate}
	\item 云端不存在请求歌曲(请求不合法):向客户端发回不合法回应,指出该歌曲不存在
	\item 播放流时网络中断:播放进行到已缓存的最后位置停止,显示"网络已中断"信息
\end{enumerate}


\paragraph{输出}

输出目标:客户端
输出内容:流/媒体文件

只要请求合法,客户端总能获得一个媒体文件/流,客户端使用相应的api播放这个媒体文件/流


\subsection{R.INTF.CLD.003 收藏}

\paragraph{介绍}

收藏功能是云音乐系统的基本功能之一.用户通过收藏功能,可以对歌曲/专辑/歌单/歌手进行标记,可以在以后快速的找到相关信息.用户的收藏信息是用户个性化的一部分,其数据将用于推荐功能中.同时,用户的收藏信息也将保存在云端,和客户端保持同步.对于收藏功能,在这里以歌曲为进行介绍,对于专辑/歌单/歌手的收藏,其行为表现与歌曲是一致的,对于歌曲独有的特性,会做出重点说明.

收藏操作是一个同步操作,要求用户在联网的状态下完成,否则视为一个异常进行处理.

\paragraph{输入}

输入来源:客户端

输入内容:歌曲+目标歌单

在客户端界面,用户可以点击歌曲项的菜单按钮进行收藏操作,用户需要手动选择希望保存到的歌单.对于"我最喜爱的歌曲"可以直接点击歌曲项上的快捷按钮,这时默认目标歌单为:我最喜爱的歌曲

\paragraph{处理}

有效性检测:

有效性检测包含歌曲和歌单两个有效性的检测:
歌曲有效性:目标歌曲必须在云端曲库中
歌单有效性:目标歌单必须在云端中并且所有者为发出请求的用户
以上两个任何一个无效都认为该次请求无效,返回无效的响应

操作时序:
\begin{enumerate}
	\item 用户在客户端进行收藏操作,客户端封装呈请求发送到云端
	\item 云端进行有效性检测,如果请求被检测为无效的,则返回给客户端无效的响应;否则,进入下一步.
	\item 系统将目标歌曲加入到目标歌单列表中,并返回给客户端 操作成功 的相应.
	\item 客户端收到相应,并在本地对数据库进行更新,更新缓存在本地的歌单列表
\end{enumerate}


异常处理:
\begin{enumerate}
	\item 在发送请求时网络中断:本次收藏操作无效,并显示错误信息.
	\item  无法收到云端发回的响应:令客户端不对歌单进行更新,直到下一次与云端同步.
\end{enumerate}


\paragraph{输出}

输出目标:客户端
输出内容:来自云端的请求响应

客户端收到云端的请求响应后,进行相应的本地歌单更新操作

\subsection{R.INTF.CLD.004 客户端}

客户端是用户与本系统交互的重要途径。本客户端具体实现是一个安卓系统上的app应用程序。客户端提供了用户完成其他需求的接口。用户可以简单的使用这个app程序而完成搜索、播放音乐、收藏音乐、使用推荐系统、完善个人信息等其他操作。

\subsubsection{需求1:前端设计}

\paragraph{介绍}

前端ui是客户端提供给用户的接口,因此前端的设计要有足够的逻辑性和方便性,这样可以方便用户进行各种操作,诸如搜索、播放音乐、收藏音乐、使用推荐系统、完善个人信息等。这些操作在前端中对应的操作应当是简洁明了的,这样才能给用户良好的体验。同时,前端还为后端提供了标准化的接口,以告知后端用户进行了哪些操作;这个接口也可以让后端将计算结果和服务器中下载的内容有一个接口用以反馈给用户。这些接口的定义应当具有足够的向上兼容性,以满足之后几代的客户端都可以使用类似的接口进行代码操作。

\paragraph{输入}

前端输入数据包括用户的操作和后端发送来的处理后的数据

\paragraph{处理}

对于这些数据,前端起到转发的作用。将用户操作数据直接发送给后端,将后端数据直接发送到屏幕显示

\paragraph{输出}

屏幕显示和后端接口


\subsubsection{需求2:后端设计}

\paragraph{介绍}

后端实现了各种后台运算和对服务器远程连接的支持。后台运算的需求包括可以将音乐文件高质量的解析出来并且调用系统的音效设备将音乐播放出来;用户使用此系统听音乐时后台不断进行这样的操作,这要求后端的音乐解析算法在保证解析质量的前提下足够高效。后端同时也要实现与服务器的通讯。后端应当实时感知系统的网络状态,在网络状态符合要求的情况下,后端将客户搜索歌曲、下载歌曲、填写个人数据等请求发送给服务器,并且接受服务器的回应。当接受到服务器压缩发送的信息后,后端对这些信息适当操作使其成为能够被前端解析的数据,并发送给前端。

\paragraph{输入}

前端发来的用户操作数据、
服务器发来的数据

\paragraph{处理}

用户操作数据作为激励信号,使后端运行对应的函数或者过程。
服务器发来的数据将在后端被解析,使之成为能够被前端识别的数据格式

\paragraph{输出}

传送到服务器的数据包。传送到前端的可以直接使用的数据

\subsection{R.INTF.CLD.005 个人信息}
个人信息由用户填写或者生成并上传服务器。采集这些信息有助于我们更好的服务用户,并且对用户提供有针对性的高质量服务。
\subsubsection{需求1:用户授权的需求}
\paragraph{介绍}
此系统采集用户信息以完善服务的过程将采集用户的一部分隐私数据,因此首先的需求是要有用户的授权。这里采用让用户使用此系统前首先确认同意此系统的服务协议的方式获得用户的授权。
\paragraph{输入}
输入来源:
用户同意用户协议的操作

数量和度量单位:
一次用户操作

时间要求:
无

包含精度和容忍度的有效输入范围:
要求精度高,即没有确认的用户不会因为误报而造成后台以为其同意的错误
\paragraph{处理}
服务器得到用户授权同意的信息后将在服务器中创建对应的账户,记录用户的信息,提供个性化服务。
\paragraph{输出}
开始提供个性化服务

\subsubsection{需求2:用户登陆信息}
\paragraph{介绍}
用户登陆信息是用于识别用户身份认证的信息。包括用户名和密码。

\paragraph{输入}

输入来源:
由于用户输入

数量:
仅几个字符

时间要求:
在用户可以忍受的时间内(1min)识别并确认该用户

包含精度和容忍度的有效输入范围:
精度要求极高,当用户输入有错时直接反馈错误情况。

\paragraph{处理}
用户名的需求是可替代性和冗余性。比如用户邮箱、音乐账号、手机号、QQ号这些信息都可以作为同一用户实体的不同用户名。我们要妥善管理这些信息,使同一用户的不同登陆信息能够识别出同一用户。而密码则是确保用户为本人操作。此云音乐系统中只需要第一次登陆的时候输入密码,此后每次登陆的时候就有记住密码的功能。这一需求是将密码以一个加密的存储在本地。密码同时也有保密性的需求,我们的数据中心服务器、用户本地都不会存储明文密码,只会存储加密后的密文。同时密码的传输会使用SSH加密传输,来保证密码不被窃取。

用户兴趣信息由用户填写或者生成。由用户填写的部分包括用户的年龄、性别、学历、喜爱的歌手等。而由用户生成的信息包括用户听歌记录、收藏的歌单、关注的用户等。这些用户的兴趣信息主要是用来为本系统的个性化推荐系统服务的。考虑到这些信息也涉及用户的部分隐私,此类信息传输时依然使用加密方式传输。而这些信息在中心服务器使用推荐算法分析时,将首先被去标签化处理。
这些信息包括和用户兴趣信息这两个部分。

\subsection{R.INTF.CLD.006 动态主页}
动态主页实时反映用户的个人信息,包括听歌记录、兴趣记录,也包括用户自定义的信息,如用户名、个性签名、头像等。

\subsubsection{需求1:实时性}
\paragraph{介绍}
实时性的需求要求动态主页能够实时更新,反应出用户的最新兴趣情况和使用记录。
\paragraph{输入}

输入来源:
实时性的更新数据输入来自软件的实时追踪用户使用情况和实时向服务器发送用户信息。这样才服务器得到获取动态主页去请求时,服务器始终能提供最新版本的动态主页数据。

数量:
若实时采集这些数据,其数量将十分庞大。而本系统将在客户端首先进行特征提取和压缩的操作,因此反馈到服务器端的数据量将会是很小的,一般只有几kB的数量。

度量单位:
实时性的度量单位是更改发生后要延迟才能体现在动态主页上。

时间要求:
实时性本身就是对时间的要求。

包含精度和容忍度的有效输入范围:
对于用户填写的内容。实时性要求精度高。而对于用户生成的内容,这些内容反应是是用户的兴趣变化实时性精度要求不高。

\paragraph{处理}
服务器端会根据这些信息更新该用户在服务器端的信息。并且在有新的访问信息的请求时提供新版本的信息

\paragraph{输出}
用户信息的实时变化。




%--------------------------------------------------------------------

\subsection{R.INTF.CLD.007:推荐功能}
\subsubsection{介绍}
本云播放器平台的推荐功能主要由两个模块共同构成,实时动作采集系统和用户评估及推荐算法。

\paragraph{1.实时动作记录采集系统}

当用户在客户端上进行动作,如收藏、评论,搜索、关注等,将这些动作实时传回云端后台加以累积记录,建立用户动作记录数据库,但记录用户的所有动作的传输存储代价太大,采用在线强化学习的方式,利用记录用户的属性偏好向量来代替所有用户动作记录,每次将动作传回后端,对该向量进行修改,后丢弃该条用户动作记录。

\paragraph{2.用户评估和推荐算法。}

根据音乐鉴赏知识划分好风格偏好向量,根据实时动作记录采集系统采集的用户动作对用户属性和偏好进行评估,建立用户属性向量,根据客户端传来的动作信息,对利用学习算法用户的属性向量进行修改,并根据实时变化的用户属性,结合推荐算法推荐用户最有可能喜欢的曲风的音乐。

\subsubsection{输入}
A.输入来源:\  客户端用户动作和用户属性向量数据库中的数据

数量:\  1或多

度量单位:\ 条数

时间要求:\ 300s以内

包含精度和容忍的有效输入范围:\ 所有有效动作,能够允许少量的数据传输丢失率

B.客户端在捕捉到动作时候 加入$Send\_Message\_toBack()$
\subsubsection{处理}
1.输入数据的有效性检测:判断是否为有效动作,即是否在确定的编号动作序列中。

2.事件顺序
\begin{itemize}
	
	
	\item 判断该动作是否有效
	\item 利用强化学习推荐算法算出对用户属性向量的修正
	\item 修改用户属性向量
\end{itemize}

假定收藏的动作编号为3,比如当用户编号A收藏了歌单编号B,AB3组合为该条动作信息传回云端。    

假设用户A的属性向量在某一时刻为{x1,x2,x3……},当动作AB3传回云端,根据强化学习评估算法用AB3跟新x1,x2,x3……,然后利用此用户属性判断A离划分好的风格偏好向量的距离,进而判断用户的喜好。

3.对异常的处理:

通信失败:\ 考虑到大量的数据传递的代价,和该功能本身大量数据长期需求,可以允许小量的通信失败,但达到阈值之后,要求重发和通知管理员。

错误处理:\ 当计算错误时,保留原有值的可靠能用。

4.判断修改后的用户属性向量是否在预先划分好的有效值以内。

\paragraph{输出}
1.输出到何处:\ 用户数据向量数据库

2.数量:\ 1条

3.度量单位:\ 条

4.对非法值得处理:\  可容忍之前的可用可靠数据,所以在这里对非法值的处理是丢弃非法值。

5.错误信息:\ 错误信息导致前数据关系建立失败。



\subsection{R.INTF.CLD.008:用户信息管理}
\subsubsection{介绍}

用户信息包括:头像,昵称,收藏和建立的歌单歌曲,关注与粉丝关系等。我们需要为这些信息建立信息管理系统,从而进行处理和维护。

针对这些信息建立本地存储和云端存储的协同方式,目的在于云端存储主数据,本地存储辅助用于优化用户体验和减小开销代价,对于用户信息管理,在云端建立用户成员数据库,并存储以上的用户信息,构建恰当的关系结构,存放与维护各种用户关系属性。当发生收藏、关注等动作时,客户端申请修改查询云端数据库的条目从而对用户数据库进行维护。

个人信息管理针对本地存储的用户信息,如头像、昵称等采用本地缓存的方式,使得用户不需要每次都对这些反复加载的信息进行处理,优化用户体验,当头像修改后,云端提示客户端头像等信息的变化,从而进行重新加载。
\subsubsection{输入}
输入来源:\ 客户端的建立删除等动作

数量:\ 1或多

度量单位:\ 条数

时间要求: \ 0.5s内

包含精度和容忍的有效输入范围:\ 不允许非法操作,并且不包容通信失败

客户端在捕捉到动作时候 加入$Send\_Message\_toBack()$
\subsubsection{处理}

1.输入数据的有效性:

判断动作是否在规定的合法动作范围之内

2.事件顺序:

根据动作进行对用户数据库的增删改查

同步到备份数据库

判断该动作所产生结果是否为可缓存结果,告知客户端

3.对错误的处理:

溢出:新分配空间和报错通知管理员。

通信失败:要求重新建立通信,否则该动作失效。

错误处理:使用备份的数据库恢复到之前的状态,重新处理。

\subsubsection{输出}
输出位置:用户信息管理数据库和客户端

数量:2

度量单位:条

包含精度和容忍度的有效输出:

不能容忍通信失败等问题带来的错误信息,当检测到发生错误时,要求重新建立。

有效输出为与定义阈值范围匹配的数据库条目。

\section{条件与限制}
\subsection{开发平台与工具}
我们使用Windows10作为主要的系统开发平台,并且使用谷歌官方推荐的Android studio作为主要的开发工具,租用腾讯的提供的虚拟主机搭建服务器后台。

\subsection{软件开发生命周期模型}
我们采用瀑布模型作为软件生命周期模型,因为瀑布模型适用于需求比较固定的情形,并且实行起来较为简单。  

\subsection{法律}
我们提供的这些云音乐资源有可能会侵犯那些著作者的版权,并且为那些提供正版云音乐的的内容提供商的利益造成一定的损害。因此为了不侵犯相关知识产权,我们只提供著作权法规定的音乐著作权保护期外的音乐作品和经过用户上传的且用户确保具有版权的音乐作品。

著作权法规定的音乐著作权的保护期指的是音乐作品的词曲作者、改编、翻译等创作者对其创作的音乐作品享有专有权的保护期限。保护期限截止于作者死亡后第50年的12月31日。合作作品截止于最后死亡的作者死亡后第50年的12月31日。过了保护期的音乐作品可以免费使用,但作者的署名权、保护作品的完整权、修改权等人身权永远受保护


\subsection{技术}
我们目前所学的知识比较浅薄,许多Android开发的知识并没有学习到或者掌握到,我们也缺少UI设计师,因此在软件开发的过程中可能会遇到各种各样的难题,因此许多问题我们会采用别人已经写好的发布到github上面的框架来实现我们想要实现的功能。  

\subsection{经费}
开发初期,我们的经费是比较少的,比如说租用虚拟主机的费用以及进行市场调研的开支,对于我们这样一群学生来说也是一笔比较大的负担。



