\chapter{软件质量特性}


本章说明项目其他的质量特性。


\section{可用性}
本云音乐播放器依赖于数据中心硬件服务器和手机操作系统客户端应用。
由于数据中心硬件的原因可能出现一些问题(可能性非常小),所以需要定期维护数据中心服务器。
同时,用户应用也需要定期更新,以保证对最新系统的兼容,并且满足用户提出的新需求。


\section{正确性}
软件能够正确运行所要求的功能。本云音乐播放器的客户端可以正确响应用户的操作,并且完成与后端的数据库的交互。
为了保证软件的正确性,我们采用用户报告差错情况,并按照标准的时间周期(一个月)进行统计计数,用来评价软件的正确性。
当采集到可以复现的错误时认为出现了bug,将在下个月的补丁中做出修复。
我们的目标是每月用户报告错误数量小于50个.同时每个更新的版本后有效错误数量减少。

\section{灵活性}
云音乐播放器能够实时根据网路信息更新音乐数据库,在版权允许范围
尽最大可能完善音乐数据库,实时推荐功能,实时根据学习推荐算法计算个体用户偏好,并且实时对个体用户的动作做出调整,推荐给软件产品用户。
我们音乐数据库的目标在24小时内,在版权允许范围内同步更新各主流音乐平台的音乐,并为之建立和修改新的推荐歌单。该软件平台为了提升竞争力需要较强的灵活性。

\section{交互工作能力}
软件系统应当具有良好的交互工作能力。用户能够在良好的交互界面下进行操作。这一能力具体实现包括以下几个方面:

当用户进行不可逆操作的时候有明确的提示:比如用户选择删除歌单或者本地数据时都有明确的再次确认,以防止错误操作。

用户常用功能易于操作:一些常用功能,如收藏歌单、搜索音乐、查看推荐等操作在主界面有清晰的入口,用户可以方便的进入。


\section{易学习性}
用户在第一次使用本地客户端时,应当出现简短的界面介绍,使用户可以了解界面上几个图标代表的功能范围,并指示用户可以通过拖拽拉出侧边栏(用户信息界面).

在用户了解图标的功能范围时,应当可以自己直接找到常用功能的位置,即可以了解到收藏歌曲,创建歌单,本地音乐,动态主页的位置.

在三次使用后,90\%以上的用户应当可以直接找到自己需要功能的位置

\section{易用性}
用户每次打开客户端,应首先进入动态首页,能在1步内找到推荐歌单,私人FM功能.考虑到用户最常进入收藏歌单中,其应当能在3步之内进入,并且不需要进行拖拽等复杂操作.

用户的个人信息界面应当可以从动态主页/个人音乐界面直接进入,且默认情况下应当隐藏.

\section{可靠性}
我们的云端数据库是双重备份,分为主数据库和备份数据库,两个备份数据库之间并尽量保持一致,这就保证当主数据库或备份数据库一方发生偏差时可以用另一方的数据对其进行校正。
二者同时出偏差的概率极低,所以用双重备份的方式能很大程度地保护服务端数据、用户数据、交互操作信息等。


\section{可移植性}
可移植性指软件产品从一种环境迁移到另一种环境的能力。
本云音乐播放器的用户端需要良好的可移植性。为了做到这一点,我们基于谷歌的安卓系统进行开发,开发时代码尽量符合安卓开发规范。这样可以保证在大多数安卓平台上都可以兼容运行。
同时我们会将客户端发布为apk文件,用户可以实现一键安装。而用户的数据在云端都有备份,所以只需要用户登陆,就能实现数据的同步。


\section{可重用性}
本云音乐系统需要实现版本的更新。因此对代码的可重用性提出了需求。我们希望每次版本更新时能尽量多的使用原有代码。

为了实现可重用性,我们在代码设计的时候就对代码根据不同的功能将代码分成不同的部分,每个部分都有明确的接口定义。而底层功能的实现对于更高层来说是透明的。

\section{鲁棒性}
鲁棒性对客户端和服务器端都提出了了需求。

对于客户端来书,我们要求对于异常操作要有友好的响应。比如没有联网时我们依然可以播放已经缓存的歌曲。

对于服务器端,我们要求服务器可以较长时间无错误的运行。在高峰时段,服务器会限制每个用户的带宽,以保证每个用户都得到响应。

\section{易学习性}

用户在第一次使用本地客户端时,应当出现简短的界面介绍,使用户可以了解界面上几个图标代表的功能范围,并指示用户可以通过拖拽拉出侧边栏(用户信息界面).

在用户了解图标的功能范围时,应当可以自己直接找到常用功能的位置,即可以了解到收藏歌曲,创建歌单,本地音乐,动态主页的位置.

在三次使用后,90\%以上的用户应当可以直接找到自己需要功能的位置

\section{易用性}
用户每次打开客户端,应首先进入动态首页,能在1步内找到推荐歌单,私人FM功能.考虑到用户最常进入收藏歌单中,其应当能在3步之内进入,并且不需要进行拖拽等复杂操作.

用户的个人信息界面应当可以从动态主页/个人音乐界面直接进入,且默认情况下应当隐藏.


